\chapter{Fundamentação Teórica}
\label{cap:fundamentacao-teorica}

% Alguns autores preferem fazer uma ``fundamentação teórica'' no segundo capítulo, outros, preferem fazer uma ``revisão da literatura''. Entretanto, isto é particular de cada trabalho e o autor deve escolher o título mais adequado para o capítulo. Consultar o orientador é importante para determinar o título apropriado.

% Evite começar da seção secundária, ou seja, não passe direto do título do capítulo para o título da seção secundária. Escreva um texto para introduzir as seções subsequentes. Lembre-se de utilizar primeira letra maiúscula quando estiver se referindo a um objeto com numeração específica como capítulo, seção, subseção, figura, tabela, quadro, equação, normalmente, se escreve a primeira letra maiúscula da palavra do objeto seguido do \textit{label}. Por exemplo, a Seção \ref{sec:citacoes} explica como fazer citações bibliográficas. Observe no código fonte deste texto como foi feita a referência cruzada. Isso permite enumerar a seção do modo automático o que facilita caso novas seções sejam criadas.  
% Esta frase mostra como citar um livro sobre descargas atmosféricas \cite{rakov2003lightning}. Também podem ser citados \textit{sites} como \citeonline{elat2015densidade}. Você precisa escrever o código da referência no arquivo "referencia.bib" dentro da pasta "elementos-pos-textuais". Veja esse, onde estão alguns exemplos que já foram testados.   

Nesta seção serão abordandos os assuntos e conceitos utilizados atualmente para soluções de melhoria de bem-estar da população, 
como: cidades inteligentes, gestão de ativos públicos, zeladoria urbana, e os algoritmos de busca. 

\section{Cidades inteligentes}
O conceito de cidades inteligentes (do inglês, \textit{smart cities}) é definido como um ambiente 
urbano que utiliza ferramentas de tecnologia da informação e comunicação (TIC) para melhorar a 
infraestrutura urbana, a utilização inteligente dos recursos, a mobilidade, a governança e os 
serviços públicos, tendo como principal beneficiado o cidadão \cite{WEISS2015}. Embora o conceito 
de cidades inteligentes seja amplamente implementado em grandes centros urbanos e metrópoles, 
é possível realizar sua adaptação a pequenos e médios municípios brasileiros. O uso de tecnologias 
em tempo real, automação de processos e a participação ativa da população possibilitam a 
adaptação e a melhoria da qualidade da gestão pública local \cite{SANTIAGO2023}.

As cidades inteligentes estruturam-se em múltiplos eixos de atuação que, integrados, promovem 
o desenvolvimento urbano sustentável e centrado no cidadão. Esses eixos abrangem diferentes 
dimensões da vida urbana e demandam soluções tecnológicas específicas para enfrentar os 
desafios contemporâneos das cidades \cite{CARVALHO2024}.

\subsection{Eixos das Cidades Inteligentes}

Nessa seção, serão abordados os eixos de atuação das cidades inteligentes, sendo eles a 
governança inteligente, a mobilidade urbana inteligente e o urbanismo e planejamento urbano.

\subsubsection{Governança Inteligente}

A governança inteligente refere-se à aplicação de tecnologias digitais para promover a 
participação cidadã, transparência administrativa e eficiência na gestão pública. Neste eixo, 
os sistemas de informação permitem que gestores públicos tomem decisões baseadas em dados, 
realizem o monitoramento de serviços em tempo real e promovam a interação direta com a 
população \cite{WEISS2015}. Ferramentas como \textit{dashboards} interativos e plataformas de dados 
abertos são exemplos de aplicações que facilitam o acesso à informação e fortalecem a 
democracia participativa \cite{DIAS2019}. A gestão orientada por dados contribui para a 
otimização de recursos públicos e para o planejamento estratégico de longo prazo, especialmente 
em contextos de restrição orçamentária.

\subsubsection{Mobilidade Urbana Inteligente}

A mobilidade urbana inteligente integra sistemas de transporte, tecnologias de 
geolocalização e análise de dados para otimizar o deslocamento de pessoas e bens nas cidades. 
Sistemas Inteligentes de Transporte (\gls{ITS}) 
utilizam sensores, câmeras e algoritmos de análise para monitorar o fluxo de tráfego, identificar 
congestionamentos e propor rotas alternativas \cite{COSTA2023}. Além disso, soluções de 
estacionamento inteligente baseadas em sistemas ciber-físicos e agentes inteligentes contribuem 
para a redução do tempo de busca por vagas e diminuição da poluição urbana \cite{BOTELHO2019}. 
A aplicação de técnicas de aprendizado de máquina permite a análise de padrões espaço-temporais 
da mobilidade, auxiliando no planejamento de intervenções estruturantes no sistema viário 
\cite{MELONIO2021}.

\subsubsection{Urbanismo e Planejamento Urbano}

O eixo de urbanismo inteligente envolve o uso de geotecnologias, modelagem de informações 
urbanas e análise espacial para o planejamento e gestão do território urbano. A adoção de 
metodologias como \gls{CIM} permite a integração de dados 
geoespaciais, infraestrutura urbana e informações socioeconômicas em modelos tridimensionais 
que auxiliam gestores no planejamento territorial e na tomada de decisões \cite{SANTIAGO2023}. 
Aplicações \textit{mobile} para relato de problemas urbanos por parte da população, integradas a sistemas 
de geolocalização, possibilitam o mapeamento colaborativo de demandas e a priorização de 
intervenções no espaço urbano \cite{SANTOS2019}. A simulação urbana integrada a sistemas 
multi-agentes e Internet das Coisas (\gls{IoT}) permite testar cenários de desenvolvimento urbano 
antes de sua implementação física \cite{CASTRO2020}. 


\section{Gestão de ativos públicos}

Os ativos públicos são recursos, direitos e serviços que são acessíveis e disponíveis para 
todos os membros de uma sociedade, com diversas finalidades, tais como: infraestrutura, serviços 
essenciais, bens comuns, patrimônio cultural e recursos naturais. A gestão de ativos públicos 
consiste no conjunto de práticas e processos que visam controlar, inventariar, manter e planejar 
a utilização desses recursos de forma eficiente e sustentável ao longo de seu ciclo de vida.
% conforme ilustrado na Figura \ref{fig:ciclo-vida-ativos} \cite{STORCK2017}.

% \begin{figure}[h!]
%     \centering
%     \captionsetup{width=12cm}
%     \Caption{\label{fig:ciclo-vida-ativos} Diagrama do ciclo de vida da gestão de ativos públicos}
%     \UFCfig{}{
%         \includegraphics[width=12cm]{figuras/qp1}
%     }{
%         \Fonte{O autor (2025).}
%     }
% \end{figure}

No contexto das cidades inteligentes, a gestão de ativos públicos torna-se estratégica para 
garantir a qualidade dos serviços urbanos e a otimização dos recursos públicos. O gerenciamento 
eficaz desses ativos demanda sistemas de informação capazes de permitir o acompanhamento em 
tempo real, com dados precisos de localização, estado de conservação e histórico de ações 
realizadas. A integração de tecnologias de geolocalização, sensores \gls{IoT} e plataformas de 
gerenciamento permite que gestores públicos tenham visibilidade completa sobre o patrimônio 
municipal \cite{SANTIAGO2023}.

Soluções informatizadas de gestão de ativos reduzem erros operacionais, previnem 
retrabalhos e promovem melhor uso dos recursos públicos \cite{STORCK2017}. Isso é 
particularmente importante em contextos onde há limitação orçamentária, como em municípios 
de pequeno e médio porte. A implementação de sistemas integrados de gestão possibilita a 
criação de inventários digitais georreferenciados, o planejamento preventivo de manutenções 
e a geração de indicadores de desempenho para a tomada de decisão baseada em evidências.

Além disso, a gestão de ativos públicos no contexto de cidades inteligentes permite a 
integração com sistemas de participação cidadã, onde a população pode reportar problemas 
relacionados à infraestrutura urbana através de aplicativos móveis \cite{SANTOS2019}. Essa 
integração entre gestão de ativos e participação popular fortalece a governança inteligente e 
contribui para a melhoria contínua dos serviços públicos, promovendo maior transparência e 
efetividade nas ações do poder público \cite{OLIVEIRA2020}.

\section{Zeladoria Urbana}

A zeladoria urbana aborda um conjunto de atividades que visam manter em boas condições 
de uso os ambientes públicos para a população, realizando serviços essenciais de manutenção e 
conservação, tais como: manutenção de ruas, poda de árvores, limpeza de vias, conservação de 
praças e calçadas. Em cidades de pequeno porte, especialmente no interior dos estados do 
Nordeste brasileiro, é possível encontrar desafios como: limitações operacionais, de 
infraestrutura e recursos humanos especializados \cite{SANTOS2019}. 

Os processos nesses contextos costumam não ser automatizados e sistematizados, o que 
compromete seriamente o planejamento, a priorização das demandas, a transparência das ações 
dos gestores e pode propiciar o mau uso de verba pública. A implementação de sistemas digitais 
de gestão de demandas de zeladoria urbana, integrados a tecnologias de geolocalização e 
priorização inteligente, pode contribuir significativamente para a eficiência operacional e para 
a melhoria da qualidade dos serviços prestados à população \cite{OLIVEIRA2020}.

Aplicativos móveis para reporte de problemas urbanos, como buracos em vias, iluminação 
pública defeituosa e lixo acumulado, têm se mostrado eficazes na mediação entre cidadãos e 
gestores públicos \cite{SANTOS2019}. Quando integrados a sistemas de gestão de ativos e 
algoritmos de otimização de rotas, esses aplicativos possibilitam não apenas o registro das 
demandas, mas também o planejamento eficiente das equipes de manutenção, a priorização 
baseada em critérios objetivos e o acompanhamento transparente do andamento das solicitações.


\section{Métodos de Otimização}

Problemas de otimização são encontrados em diversas aplicações 
práticas, por exemplo o planejamento de rotas. Esses problemas 
buscam encontrar uma solução factível dentre um conjunto de soluções possíveis, de acordo com 
determinado critério de otimização \cite{CORMEN2012}. Os métodos para resolver problemas 
de otimização podem ser classificados em diferentes categorias, cada uma com características, 
vantagens e limitações específicas.

\subsection{Métodos Exatos}

Métodos exatos são aqueles que garantem encontrar a solução ótima para um problema de 
otimização. Esses métodos exploram sistematicamente o espaço de soluções, avaliando todas 
as possibilidades ou utilizando técnicas de poda para eliminar soluções que comprovadamente 
não podem ser ótimas \cite{CORMEN2012}.

Entre as principais técnicas exatas, destacam-se:

\begin{itemize}
    \item \textbf{Enumeração Completa:} Consiste em avaliar todas as soluções possíveis e 
    selecionar a melhor. Embora garanta a solução ótima, torna-se impraticável para problemas 
    com grande espaço de busca, pois geralmente possuem um crescimento exponencial do número de soluções.
    
    \item \textbf{Programação Dinâmica:} Técnica que resolve problemas complexos dividindo-os 
    em subproblemas menores e armazenando os resultados desses subproblemas para evitar 
    recálculos \cite{CORMEN2012}. É aplicável quando o problema possui subestrutura ótima 
    (a solução ótima contém soluções ótimas dos subproblemas) e subproblemas sobrepostos.
    
    \item \textbf{\textit{Branch-and-Bound}:} Método que particiona o espaço de soluções em 
    subconjuntos (ramificação) e utiliza limitantes para podar ramos que não podem conter a 
    solução ótima. É amplamente utilizado em problemas de otimização combinatória, como o 
    problema do caixeiro viajante.
    
    \item \textbf{Programação Linear e Inteira:} Técnicas matemáticas que formulam o problema 
    de otimização como um conjunto de equações e inequações lineares. Enquanto a programação 
    linear lida com variáveis contínuas, a programação inteira trata de variáveis discretas, sendo 
    esta última geralmente mais complexa computacionalmente.
\end{itemize}

A principal limitação dos métodos exatos é a complexidade computacional. Parte dos problemas 
de otimização pertencem à classe NP-difícil, onde não se conhece algoritmo que encontre a 
solução ótima em tempo polinomial. Para instâncias de grande porte, o tempo necessário para 
obter a solução ótima pode ser impraticável, motivando o uso de métodos aproximados 
\cite{CORMEN2012}.

\subsection{Heurísticas}

Heurísticas são métodos que buscam encontrar boas soluções em tempo computacional 
razoável, sem necessariamente garantir a otimalidade. Uma heurística é uma regra prática, 
estratégia ou técnica que simplifica a resolução de problemas complexos \cite{RUSSELL2013}.

As principais características das heurísticas incluem:

\begin{itemize}
    \item \textbf{Eficiência computacional:} Geralmente possuem complexidade de tempo 
    polinomial, permitindo resolver instâncias grandes em tempo razoável.
    
    \item \textbf{Qualidade de solução:} Embora não garantam a solução ótima, frequentemente 
    produzem soluções de boa qualidade, próximas ao ótimo.
    
    \item \textbf{Especificidade:} Parte das heurísticas são desenvolvidas explorando características 
    específicas do problema, tornando-as eficazes para determinados tipos de instâncias.
\end{itemize}

Exemplos de heurísticas construtivas incluem o algoritmo do vizinho mais próximo para o 
problema do caixeiro viajante, onde, partindo de um vértice inicial, sempre se escolhe o vértice 
não visitado mais próximo até completar o circuito. Heurísticas de melhoria, como a busca 
local, partem de uma solução inicial e iterativamente aplicam modificações locais que melhoram 
a qualidade da solução \cite{CORMEN2012}.

\subsection{Meta-heurísticas}

Meta-heurísticas são estratégias de alto nível que guiam heurísticas subordinadas na busca 
por soluções de qualidade. Diferentemente das heurísticas clássicas, as meta-heurísticas são 
mais genéricas e podem ser aplicadas a diversos problemas de otimização com pequenas 
adaptações \cite{RUSSELL2013}.

A escolha entre métodos exatos, heurísticas ou meta-heurísticas depende do contexto da 
aplicação, considerando fatores como o tamanho da instância, requisitos de qualidade da 
solução, tempo disponível para computação e recursos computacionais \cite{CORMEN2012}.

% \subsubsection{Algoritmos Genéticos}

% Os Algoritmos Genéticos (\gls{AG}) constituem uma meta-heurística inspirada nos princípios da 
% evolução biológica, simulando o processo de seleção natural para evoluir uma população de 
% soluções candidatas ao longo de gerações sucessivas \cite{RUSSELL2013}. São particularmente 
% eficazes para problemas de otimização combinatória NP-difíceis, como o problema do caixeiro 
% viajante.

% A estrutura básica de um AG envolve os seguintes componentes principais: 
% (i) \textbf{Cromossomo} - representação de uma solução candidata, codificada como sequência 
% de genes (no TSP, uma permutação das cidades); 
% (ii) \textbf{População} - conjunto de soluções que evoluem simultaneamente; 
% (iii) \textbf{Função de Aptidão (\textit{Fitness})} - métrica que avalia a qualidade de cada 
% solução, determinando sua probabilidade de reprodução.

% O algoritmo opera através de três operadores genéticos fundamentais:

% \begin{itemize}
%     \item \textbf{Seleção:} Escolhe indivíduos para reprodução favorecendo aqueles com maior 
%     aptidão. Métodos comuns incluem seleção por torneio, roleta e ranking.
    
%     \item \textbf{Cruzamento (\textit{Crossover}):} Combina características de dois 
%     cromossomos pais para gerar descendentes. Para problemas de permutação como o \gls{TSP}, 
%     utiliza-se técnicas como o Cruzamento de Ordem (\gls{OX}), que preserva a ordem relativa dos 
%     elementos. Outras variantes incluem cruzamento de um ponto, dois pontos e uniforme.
    
%     \item \textbf{Mutação:} Introduz pequenas alterações aleatórias nos cromossomos (como 
%     troca de posições ou inversão de segmentos), mantendo a diversidade genética e evitando 
%     convergência prematura para ótimos locais.
% \end{itemize}

% O ciclo evolutivo segue a sequência: (1) Inicialização de população aleatória; 
% (2) Avaliação da aptidão de cada indivíduo; (3) Seleção dos melhores para reprodução; 
% (4) Aplicação de cruzamento e mutação para gerar nova geração; (5) Substituição da população 
% antiga, frequentemente mantendo os melhores indivíduos (elitismo). O processo se repete até 
% atingir um critério de parada, como número máximo de gerações ou qualidade de solução 
% satisfatória.

% No contexto de otimização de rotas, os AGs permitem encontrar 
% soluções de boa qualidade em tempo computacional razoável, mesmo para instâncias com grande 
% número de pontos, onde métodos exatos seriam inviáveis. Suas principais vantagens incluem: 
% exploração simultânea do espaço de soluções, robustez a ótimos locais, flexibilidade para 
% diferentes representações e capacidade de paralelização \cite{CORMEN2012}.

\section{Teoria dos Grafos}

Um grafo é uma estrutura matemática utilizada para modelar relações entre objetos, sendo 
amplamente empregado na ciência da computação para representar redes, mapas, circuitos e 
diversos outros sistemas \cite{CORMEN2012}. Formalmente, um grafo $G$ é definido pelo par 
ordenado $G = (V, E)$, onde:

\begin{itemize}
    \item \textbf{Vértices (ou Nós):} $V$ é um conjunto finito não-vazio de elementos chamados 
    vértices. Cada vértice representa uma entidade do sistema modelado. Por exemplo, em um 
    grafo representando uma rede de cidades, cada vértice corresponde a uma cidade.
    
    \item \textbf{Arestas:} $E$ é um conjunto de pares de vértices, denominados arestas, que 
    representam as conexões ou relações entre os vértices. Uma aresta $e \in E$ conecta dois 
    vértices, podendo ser denotada como $(u, v)$, onde $u, v \in V$.
\end{itemize}

Os grafos podem ser classificados segundo diferentes critérios \cite{CORMEN2012}:

\subsection{Grafos Direcionados e Não-Direcionados}

\begin{itemize}
    \item \textbf{Grafo Não-Direcionado:} As arestas não possuem direção, ou seja, a aresta 
    $(u, v)$ é equivalente a $(v, u)$. Representa relações simétricas, como estradas de mão 
    dupla entre cidades.
    
    \item \textbf{Grafo Direcionado (Dígrafo):} As arestas possuem direção, representadas por 
    setas. A aresta $(u, v)$ indica uma conexão de $u$ para $v$, mas não necessariamente de 
    $v$ para $u$. Útil para modelar ruas de mão única, dependências entre tarefas, etc.
\end{itemize}

% A Figura \ref{fig:grafo-nao-direcionado} ilustra duas representações de um grafo não 
% direcionado: o grafo propriamente dito, sua representação por lista de adjacências e por 
% matriz de adjacências. Já a Figura \ref{fig:grafo-direcionado} apresenta as mesmas 
% representações para um grafo direcionado, evidenciando o sentido das conexões através de setas.

% \begin{figure}[h!]
%     \centering
%     \captionsetup{width=14cm}
%     \Caption{\label{fig:grafo-nao-direcionado} Duas representações de um grafo não dirigido. 
%     (a) Um grafo não dirigido G com cinco vértices e sete arestas. (b) Uma representação de G 
%     por lista de adjacências. (c) A representação de G por matriz de adjacências}
%     \UFCfig{}{
%         \includegraphics[width=14cm]{figuras/Grafo_nao_direcionado}
%     }{
%         \Fonte{\cite{CORMEN2012}.}
%     }
% \end{figure}

% \begin{figure}[h!]
%     \centering
%     \captionsetup{width=14cm}
%     \Caption{\label{fig:grafo-direcionado} Duas representações de um grafo dirigido. 
%     (a) Um grafo dirigido G com seis vértices e oito arestas. (b) Uma representação de G 
%     por lista de adjacências. (c) A representação de G por matriz de adjacências}
%     \UFCfig{}{
%         \includegraphics[width=14cm]{figuras/Grafo_direcionado}
%     }{
%         \Fonte{\cite{CORMEN2012}.}
%     }
% \end{figure}

\subsection{Grafos Ponderados}

Em muitas aplicações práticas, as arestas possuem pesos ou custos associados. Um grafo 
ponderado é aquele em que existe uma função $w: E \rightarrow \mathbb{R}$ que atribui um 
valor numérico a cada aresta \cite{CORMEN2012}. Esses pesos podem representar:

\begin{itemize}
    \item Distâncias entre localizações geográficas;
    \item Tempo de deslocamento entre pontos;
    \item Custo de transporte;
    \item Capacidade de transmissão de dados em redes.
\end{itemize}

\subsection{Conceitos Adicionais}

\begin{itemize}
    \item \textbf{Adjacência:} Dois vértices são adjacentes se existe uma aresta conectando-os.
    
    \item \textbf{Grau de um Vértice:} Em grafos não-direcionados, o grau de um vértice $v$ é 
    o número de arestas incidentes a ele. Em grafos direcionados, distingue-se entre grau de 
    entrada (arestas que chegam) e grau de saída (arestas que saem).
    
    \item \textbf{Caminho:} Sequência de vértices onde cada par consecutivo está conectado por 
    uma aresta. O comprimento de um caminho é o número de arestas que o compõem.
    
    \item \textbf{Ciclo:} Caminho que começa e termina no mesmo vértice, sem repetir arestas.
    
    \item \textbf{Grafo Conexo:} Grafo não-direcionado onde existe um caminho entre qualquer 
    par de vértices.
\end{itemize}

\subsection{Representação Computacional}

Grafos podem ser representados computacionalmente de duas formas principais \cite{CORMEN2012}:

\begin{itemize}
    \item \textbf{Matriz de Adjacência:} Matriz $A$ de dimensão $|V| \times |V|$, onde 
    $A[i][j] = 1$ se existe aresta entre os vértices $i$ e $j$, e $A[i][j] = 0$ caso contrário. 
    Para grafos ponderados, $A[i][j]$ armazena o peso da aresta. Ocupa espaço $O(V^2)$.
    
    \item \textbf{Lista de Adjacência:} Para cada vértice, mantém-se uma lista dos vértices 
    adjacentes. Mais eficiente em termos de espaço para grafos esparsos, ocupando $O(V + E)$.
\end{itemize}

Os algoritmos de busca em grafos são adotados para resolver problemas de caminho mínimo, 
conectividade, detecção de ciclos e otimização de rotas, constituindo ferramentas essenciais 
para aplicações em logística, redes de computadores e planejamento urbano \cite{CORMEN2012}.

\section{Algoritmos de Busca}
Nessa seção, seram apresentados alguns dos principais algoritmos de busca em grafos, sendo eles a Busca em Largura (BFS), 
a Busca em Profundidade (DFS), a Busca de Custo Uniforme (UCS) e o Algoritmo A* (A Estrela).

\subsection{Busca em Largura (BFS)}

A Busca em Largura \gls{BFS} explora o grafo sistematicamente 
visitando todos os vértices a uma mesma distância da origem antes de avançar para níveis mais 
profundos \cite{CORMEN2012}. O algoritmo utiliza uma estrutura de fila \gls{FIFO} para controlar a ordem de visitação dos vértices.

A complexidade de tempo da BFS é $O(V + E)$, onde $V$ é o número de vértices e $E$ é o 
número de arestas. O algoritmo garante encontrar o caminho mais curto em termos de número 
de arestas quando todas têm o mesmo peso. A BFS é particularmente útil para:

\begin{itemize}
    \item Encontrar o menor caminho em grafos não ponderados;
    \item Testar a conectividade de um grafo;
    \item Encontrar todos os vértices alcançáveis a partir de um vértice de origem;
    \item Detectar ciclos em grafos não direcionados.
\end{itemize}

A Figura \ref{fig:bfs-exemplo} ilustra o funcionamento da busca em largura em um grafo, 
mostrando a ordem de visitação dos vértices nível por nível.

\begin{figure}[h!]
    \centering
    \captionsetup{width=14cm}
    \Caption{\label{fig:bfs-exemplo} Exemplo de execução da Busca em Largura (BFS) em um grafo}
    \UFCfig{}{
        \includegraphics[width=14cm]{figuras/exemple_BFS}
    }{
        \Fonte{\cite{RUSSELL2013}.}
    }
\end{figure}

\subsection{Busca em Profundidade (DFS)}

A Busca em Profundidade \gls{DFS} percorre o grafo explorando o 
máximo possível cada caminho antes de retroceder \cite{CORMEN2012}. Pode ser implementada 
recursivamente ou utilizando uma pilha \gls{LIFO}. A \gls{DFS} também possui 
complexidade $O(V + E)$.

A \gls{DFS} é especialmente adequada para:

\begin{itemize}
    \item Detecção de ciclos em grafos direcionados e não direcionados;
    \item Ordenação topológica em grafos acíclicos direcionados (\gls{DAG});
    \item Decomposição de grafos em componentes fortemente conexos;
    \item Resolução de quebra-cabeças e problemas que exigem exploração de todas as 
    possibilidades, como labirintos.
\end{itemize}

A Figura \ref{fig:dfs-exemplo} demonstra o processo de busca em profundidade, evidenciando 
como o algoritmo explora cada ramo até seu limite antes de retroceder.

\begin{figure}[h!]
    \centering
    \captionsetup{width=14cm}
    \Caption{\label{fig:dfs-exemplo} Exemplo de execução da Busca em Profundidade (DFS) em um grafo}
    \UFCfig{}{
        \includegraphics[width=14cm]{figuras/exemple_DFS}
    }{
        \Fonte{\cite{RUSSELL2013}.}
    }
\end{figure}

\subsection{Busca de Custo Uniforme}

A Busca de Custo Uniforme \gls{UCS} é uma estratégia de busca 
que expande o nó com menor custo de caminho acumulado, sendo uma generalização da \gls{BFS} para 
grafos com arestas de diferentes custos \cite{RUSSELL2013}. Diferentemente da BFS, que trata 
todas as arestas igualmente, a UCS considera os pesos associados às arestas para determinar 
a ordem de expansão dos nós.

O algoritmo utiliza uma fila de prioridade onde os nós são ordenados pelo custo total do 
caminho da origem $g(n)$, sempre expandindo o nó com menor custo acumulado. A cada expansão, 
os custos dos nós vizinhos são atualizados se um caminho mais barato for encontrado. A UCS 
é completa e ótima, garantindo encontrar a solução de menor custo quando todos os custos das 
arestas são não-negativos \cite{RUSSELL2013}.

A complexidade de tempo e espaço da busca de custo uniforme é $O(b^{1+\lfloor C^*/\epsilon \rfloor})$, 
onde $b$ é o fator de ramificação, $C^*$ é o custo da solução ótima, e $\epsilon$ é o menor 
custo de aresta. Na prática, com implementação eficiente usando \textit{heap} binário, a complexidade 
é $O((V + E) \log V)$.

A UCS é particularmente adequada para:

\begin{itemize}
    \item Encontrar caminhos de menor custo em grafos ponderados com pesos não-negativos;
    \item Problemas de roteamento onde diferentes caminhos têm custos variados;
    \item Planejamento de rotas considerando distâncias, tempos ou custos operacionais;
    \item Situações onde não há informação heurística disponível sobre a proximidade do objetivo.
\end{itemize}

A principal diferença entre a UCS e algoritmos informados como o A* é que a UCS não utiliza 
conhecimento adicional (heurística) sobre a distância até o objetivo, expandindo nós apenas 
com base no custo acumulado desde a origem \cite{RUSSELL2013}.



\subsection{Algoritmo A* (A Estrela)}

O algoritmo A* é uma extensão da busca de custo uniforme que incorpora uma função heurística 
para guiar a busca em direção ao objetivo \cite{RUSSELL2013}. A função de avaliação é definida 
como $f(n) = g(n) + h(n)$, onde $g(n)$ é o custo do caminho da origem até o vértice $n$, e 
$h(n)$ é a heurística que estima o custo de $n$ até o objetivo.

Para que o A* garanta encontrar o caminho ótimo, a heurística deve ser admissível, ou seja, 
nunca superestimar o custo real até o objetivo. Heurísticas comuns incluem a distância 
euclidiana e a distância de Manhattan em espaços bidimensionais \cite{RUSSELL2013}.

O A* é particularmente eficiente quando existe uma boa heurística disponível, pois reduz 
significativamente o número de vértices explorados em comparação com Dijkstra. É amplamente 
utilizado em:

\begin{itemize}
    \item Jogos e simulações para movimentação de personagens e entidades;
    \item Robótica para planejamento de trajetórias;
    \item Sistemas de navegação com informação geográfica;
    \item Otimização de rotas considerando múltiplos critérios.
\end{itemize}

% A Figura \ref{fig:aestrela-exemplo} apresenta um exemplo clássico de aplicação do algoritmo A* 
% para encontrar o caminho mais curto entre cidades romenas, demonstrando como a heurística 
% guia a busca de forma eficiente.

% \begin{figure}[h!]
%     \centering
%     \captionsetup{width=14cm}
%     \Caption{\label{fig:aestrela-exemplo} Exemplo de aplicação do algoritmo A* no problema de roteamento entre cidades da Romênia}
%     \UFCfig{}{
%         \includegraphics[width=14cm]{figuras/exemple_A_Estrela(Bucareste)}
%     }{
%         \Fonte{\cite{RUSSELL2013}.}
%     }
% \end{figure}




% cidades intelihentes: 
% -eixos(governança, mobilidade, urbanismo)
% -Gestão de ativos públicos
% Otimização:
% -merodos exatos, euristica, etc.
% -algoritmos de busca




% A Figura \ref{fig:reitoria} apresenta a fotografia da reitoria da Universidade Federal do Ceará. Observe a estrutura do código para ver como inserir figuras. No título, comece especificando o tipo de figura. Por exemplo, fotografia, desenho, diagrama, fluxograma, gráfico e etc. O espaçamento entre linhas no título é de $1~pt$ (espaçamento simples), apenas a primeira letra da frase é maiúscula. As demais palavras são escritas com letra maiúsculas somente quando são nomes próprios e não há ponto final. 
    
%     As margens do título da figura são delimitadas pelo tamanho da figura. Por isso, procure ajustar o tamanho da figura para preencher a largura delimitada pelas margens esquerda e direita da página que possui $16~cm$ de largura. Não esqueça de indicar fonte da figura. O autor deve evitar deixar figuras pequenas menores do que $7~cm$ de largura.
    
%     A posição da figura deve ser o mais próximo logo após ter sido chamada no texto (a figura nunca deve aparecer antes de ter sido anunciada no texto). 
    
%     %troque h pelo b ou t para mudar a posição da figura.
%  	\begin{figure}[h!] 
%    	    \captionsetup{width=16cm}%Da mesma largura que a figura
% 		\Caption{\label{fig:reitoria} Fotografia da reitoria da Universidade Federal do Ceará}
% 		\UFCfig{}{
% 			\includegraphics[width=16cm]{figuras/exemplo-1}
% 		}{
% 			\Fonte{\citeonline{UFC2012}.}
% 		}	
% 	\end{figure}
                    

% \section{Inserindo figuras}\label{sec:figuras}
    
    
	
%     Texto1 texto texto texto texto texto texto texto texto texto texto texto texto texto texto texto texto texto texto texto texto texto texto texto texto texto texto texto texto texto texto texto texto texto texto texto texto texto texto texto texto texto texto texto texto1.

%     Texto2 texto texto texto texto texto texto texto texto texto texto texto texto texto texto texto texto texto texto. Texto texto texto texto texto texto texto texto texto texto texto texto texto texto texto texto texto texto texto2.

%     Texto3 texto texto texto texto texto texto texto texto texto texto texto texto texto texto texto texto texto texto. Texto texto texto texto texto texto texto texto texto texto texto texto texto texto texto texto texto texto texto3.

%     Texto4 texto texto texto texto texto texto texto texto texto texto texto texto texto texto texto texto texto texto. Texto texto texto texto texto texto texto texto texto texto texto texto texto texto texto texto texto texto texto4.

%     A Figura \ref{fig:sondas} Texto texto texto texto texto texto texto texto texto texto texto texto texto texto texto texto texto texto texto. Texto texto texto texto texto texto texto texto texto texto texto texto texto texto texto texto texto texto texto3.

% 	\begin{figure}[h!]
% 		\centering
% 		\captionsetup{width=14cm}%Da mesma largura que a figura
% 		\Caption{\label{fig:sondas} Gráfico da Atmosfera Superior}	
% 		\UFCfig{}{
% 			\includegraphics[width=14cm]{figuras/sondas}
% 		}{
% 			\Fonte{adaptado da \citeonline{NASA2016}.}}	
% 	\end{figure}

%     Texto5 texto texto texto texto texto texto texto texto texto texto texto texto texto texto texto texto texto texto texto texto texto texto texto texto texto texto texto texto texto texto texto texto texto texto texto texto texto texto texto texto texto texto texto texto5.

%     Texto6 texto texto texto texto texto texto texto texto texto texto texto texto texto texto texto texto texto texto texto texto texto texto texto texto texto texto texto texto texto texto texto texto texto texto texto texto texto texto texto texto texto texto texto texto5.

%     Texto7 texto texto texto texto texto texto texto texto texto texto texto texto texto texto texto texto texto texto texto texto texto texto texto texto texto texto texto texto texto texto texto texto texto texto texto texto texto texto texto texto texto texto texto texto texto texto texto texto texto texto texto texto texto texto texto texto texto texto texto texto texto texto texto6.

%     Evite terminar seções, capítulos e etc com figura. Procure escrever mais.

% \section{Inserindo tabelas}\label{sec:tabelas}
    
%     A Tabela \ref{tab:exemplo-1}... texto texto texto texto texto texto texto texto texto texto texto texto texto texto texto texto texto texto texto. Texto texto texto texto texto texto texto texto texto texto texto texto texto texto texto texto texto texto texto.
	
% 	\begin{table}[!h]
% 	\captionsetup{width=7cm}%Deixe da mesma largura que a tabela
% 	\Caption{\label{tab:exemplo-1} Um Exemplo de tabela alinhada que pode ser longa ou curta}%
% 	\IBGEtab{}{%
% 		\begin{tabular}{ccc}
% 			\toprule
% 			Nome & Nascimento & Documento \\
% 			\midrule \midrule
% 			Maria da Silva & 11/11/1111 & 111.111.111-11 \\
% 			Maria da Silva & 11/11/1111 & 111.111.111-11 \\
% 			Maria da Silva & 11/11/1111 & 111.111.111-11 \\
% 			\bottomrule
% 		\end{tabular}%
% 	}{%
% 	\Fonte{o autor.}%
% 	\Nota{esta é uma nota, que diz que os dados são baseados na
% 		regressão linear.}%
% 	\Nota[Anotações]{uma anotação adicional, seguida de várias outras.}%
%     }
%     \end{table}

% 	%\begin{table}[h!]	
% 	%	\centering
% 	%	\Caption{\label{tab:exemplo-1} Exemplo de tabela}	
% 	%	\UFCtab{}{
% 	%		\begin{tabular}{cll}
% 	%			\toprule
% 	%			Ranking & Exon Coverage & Splice Site Support \\
% 	%			\midrule \midrule
% 	%			E1 & Complete coverage by a single transcript & Both splice sites\\
% 	%			E2 & Complete coverage by more than a single transcript & Both splice sites\\
% 	%			E3 & Partial coverage & Both splice sites\\
% 	%			E4 & Partial coverage & One splice site\\
% 	%			E5 & Complete or partial coverage & No splice sites\\
% 	%			E6 & No coverage & No splice sites\\
% 	%			\bottomrule
% 	%		\end{tabular}
% 	%	}{
% 	%	\Fonte{elaborado pelo autor.}
% 	%}
% 	%\end{table}

% \subsection{Exemplo de subseção} \label{sec:ex_sec}
	
%     Texto texto texto texto texto texto texto texto texto texto texto texto texto texto texto texto texto texto texto texto texto texto texto texto texto texto texto texto texto texto texto texto texto texto texto texto texto texto texto texto texto texto texto texto texto.

%     %acrlong{DATASUS},\acrlong{DNV},\acrlong{DO},\acrlong{ESF},\acrlong{IBGE},\acrlong{MFC},\acrlong{MI},\acrlong{MS},\acrlong{NV},\acrlong{ODM},\acrlong{OI},\acrlong{OMS},\acrlong{ONU},\acrlong{PNI},\acrlong{PSF},\acrlong{RIPSA},\acrlong{RN},\acrlong{SIM},\acrlong{SINASC},\acrlong{SUS},\acrlong{TMI},\acrlong{TMMFC}


%     \begin{alineascomponto}
% 	    \item Integer non lacinia magna. Aenean tempor lorem tellus, non sodales nisl commodo ut
% 	    \item Proin mattis placerat risus sit amet laoreet. Praesent sapien arcu, maximus ac fringilla efficitur, vulputate faucibus sem. Donec aliquet velit eros, sit amet elementum dolor pharetra eget
% 	    \item Integer eget mattis libero. Praesent ex velit, pulvinar at massa vel, fermentum dictum mauris. Ut feugiat accumsan augue, et ultrices ipsum euismod vitae
% 	    \begin{subalineascomponto}
% 		    \item Integer non lacinia magna. Aenean tempor lorem tellus, non sodales nisl commodo ut
% 		    \item Proin mattis placerat risus sit amet laoreet.
% 	    \end{subalineascomponto}
%     \end{alineascomponto}

% \subsection{Uso de siglas} \label{sec:siglas}

%     Para utilizar siglas, primeiro defina a sigla no arquivo "lista-de-abreviaturas-e-siglas"~ dentro da pasta "1-pre-textuais" com o comando 
%     \begin{verbatim}
%         \newacronym{ABNT}{ABNT}{Associação Brasileira de Normas Técnicas}
%     \end{verbatim}
%     Depois chame a sigla com o comando:
%     \begin{verbatim}
%         \gls{ABNT}
%     \end{verbatim}
%     Fica assim: \gls{ABNT}. A primeira vez que o comando é usado para uma determinada sigla, aparece o significado por extenso da sigla com a sua abreviação em seguida. A partir da segunda vez que o comando para uma determinada sigla é usado, aparace apenas a sigla. Por exemplo: \gls{ABNT}.  
    
%     Veja o código fonte de outros exemplos: Teste de siglas \gls{TEST}, outros exemplos de siglas: \gls{DA}, \gls{MCEG}. 
%     Repare que sempre as siglas estão sendo definidas primeiramente no arquivo ``lista-de-abreviaturas-e-siglas''.