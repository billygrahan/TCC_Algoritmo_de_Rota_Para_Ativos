\chapter{Introdução}
\label{cap:introducao}

%Para começar a usar este \textit{template}, na plataforma \textit{ShareLatex}, vá nas opções (três barras vermelhas horizontais) no canto esquerdo superior da tela e clique em "Copiar Projeto" e dê um novo nome para o projeto. 

% Para começar a utilizar este \textit{template}, siga o tutorial clicando no seguinte \textit{link}:
% \url{https://biblioteca.ufc.br/wp-content/uploads/2015/09/tutorial-sharelatex.pdf}

% Neste \textit{template}, o autor irá encontrar diversas instruções e exemplos dos recursos do uso do \LaTeX~na plataforma \textit{Overleaf}. O \LaTeX~foi desenvolvido, inicialmente, na década de 80, por Leslie Lamport e é utilizado amplamente na produção de textos matemáticos e científicos, devido a sua alta qualidade tipográfica \cite{goossens1994latex}. 

% O \textit{ShareLatex} é uma plataforma \textit{online} que pode ser acessado por meio de qualquer navegador de internet até mesmo de um \textit{smartphone}. Essa plataforma dispensa a instalação de aplicativos no computador para desenvolver trabalhos em \LaTeX. Também, não é necessário instalar \textit{packages}, ou seja, pacotes que permitem diferentes efeitos na formatação e no visual do trabalho. Todos os \textit{packages} que este \textit{template} utiliza são encontrados \textit{online}. 

% Apresentam-se, também, neste modelo, algumas orientações de como desenvolver um trabalho acadêmico. Entretanto, este arquivo deve ser editado pelo autor de acordo com o seu trabalho sendo que a formatação já está de acordo com o aceito pela Universidade Federal do Ceará.  

% A introdução, tem como finalidade, dar ao leitor uma visão concisa do tema investigado, ressaltando-se o assunto de forma delimitada, ou seja, enquadrando-o sob a perspectiva de uma área do conhecimento, de forma que fique evidente sobre o que se está investigando; a justificativa da escolha do tema; os objetivos do trabalho; o objeto de pesquisa que será investigado. Observe que não se divide a introdução em seções, mas a mesma informa como o trabalho ao todo está organizado.





A gestão eficiente dos ativos públicos constitui um desafio constante para as administrações municipais, 
especialmente quando demanda visitas periódicas para inspeção, manutenção ou levantamento de informações. 
Esses ativos — como praças, escolas, unidades de saúde, postes de iluminação, equipamentos urbanos, entre outros — 
encontram-se com algum Problema por diferentes regiões da cidade, exigindo um planejamento logístico criterioso que minimize 
o tempo de deslocamento e os custos operacionais das equipes responsáveis.

No contexto das cidades inteligentes, a otimização de rotas torna-se fundamental para a eficiência 
operacional dos serviços públicos. A ausência de planejamento adequado resulta em desperdício de recursos, 
aumento do tempo de resposta às demandas da população e ineficiência na prestação de serviços essenciais. 
Este cenário é particularmente crítico em municípios de pequeno e médio porte, onde as limitações orçamentárias 
e de infraestrutura são mais acentuadas.

O desafio consiste em determinar a rota mais eficiente para que 
profissionais da prefeitura percorram todos os pontos de visitação necessários e retornem ao ponto de origem, 
minimizando a distância total percorrida e, consequentemente, o tempo e os custos operacionais.

\section{Objetivos}

% Diante dessa problemática, o presente trabalho propõe o desenvolvimento de uma solução algorítmica 
% baseada em técnicas heurísticas para resolver o Problema do Caixeiro Viajante (\textit{Traveling Salesman Problem} -- TSP) 
% aplicado à gestão de ativos públicos municipais. 

Diante do contexto apresentado, o objetivo geral deste trabalho consiste em desenvolver e avaliar um algoritmo 
capaz de gerar rotas otimizadas para a visitação de ativos públicos por profissionais da administração municipal, 
aplicando técnicas heurísticas para resolver o Problema do Caixeiro Viajante (\textit{Traveling Salesman Problem} -- TSP) 
adaptado à realidade da gestão pública.

\subsection{Objetivos específicos}
    Para alcançar o objetivo geral, estabelecem-se os seguintes objetivos específicos:
\begin{itemize}
    \item Realizar uma revisão sistemática da literatura para identificar os principais trabalhos relacionados 
    à otimização de rotas em contextos urbanos e cidades inteligentes, bem como as técnicas e algoritmos empregados;
    \item Identificar o Problema do Caixeiro Viajante e suas variantes, contextualizando suas 
    aplicações na gestão de ativos públicos municipais;
    \item Obter um algoritmo heurístico para geração de rotas otimizadas, 
    considerando as especificidades da gestão pública;
    \item Obter uma avaliação do desempenho do algoritmo desenvolvido por meio de análise comparativa utilizando 
    dados de ativos municipais;
    % \item Analisar a viabilidade prática da solução proposta para aplicação em prefeituras de pequeno e médio porte.
\end{itemize}

\section{Justificativa}

A relevância deste trabalho fundamenta-se em múltiplas dimensões que abrangem aspectos técnicos, econômicos e sociais. 
Do ponto de vista técnico, o Problema do Caixeiro Viajante representa um desafio clássico da ciência da computação, 
classificado como NP-difícil, cuja solução exata torna-se impraticável para instâncias de grande porte. 
A aplicação de técnicas heurísticas permite obter soluções de boa qualidade em tempo computacional razoável, 
viabilizando sua implementação em contextos práticos.

No âmbito econômico, a otimização de rotas contribui diretamente para a redução de custos operacionais 
no setor público. A economia gerada pela minimização das distâncias percorridas reflete-se na diminuição 
do consumo de combustível, na redução do desgaste dos veículos e no aproveitamento mais eficiente do 
tempo dos profissionais. Essa otimização representa uma ferramenta estratégica para maximizar a eficiência 
dos recursos disponíveis.

Sob a perspectiva social, a melhoria na eficiência operacional da administração pública impacta 
diretamente a qualidade dos serviços prestados à população. Rotas otimizadas permitem maior agilidade 
no atendimento às demandas de manutenção urbana, reduzindo o tempo de resposta e aumentando a 
capacidade de atendimento das equipes municipais. Esse ganho de eficiência fortalece a confiança 
da população nas instituições públicas e contribui para a melhoria da qualidade de vida nos centros urbanos.

Além disso, este trabalho insere-se no contexto contemporâneo das cidades inteligentes, 
onde a aplicação de tecnologias da informação e técnicas de otimização tornam-se instrumentos 
essenciais para a modernização da gestão pública. A solução proposta pode ser adaptada
e replicada em diversos municípios, especialmente aqueles de pequeno e médio porte, 
democratizando o acesso a ferramentas tecnológicas que promovam eficiência administrativa 
e transparência na gestão dos recursos públicos.

\section{Organização do Trabalho}

Este trabalho está estruturado em cinco capítulos, organizados de forma a proporcionar uma compreensão 
progressiva e sistemática do tema abordado.

O Capítulo 1 apresenta a contextualização do problema, os objetivos da pesquisa, 
a justificativa e a organização geral do trabalho, fornecendo uma visão panorâmica 
dos elementos centrais desta investigação.

O Capítulo 2 expõe a fundamentação teórica necessária para a compreensão do trabalho,
abordando os conceitos de cidades inteligentes, gestão de ativos públicos, zeladoria 
urbana, métodos de otimização e teoria dos grafos. São apresentados os principais 
algoritmos de busca e técnicas heurísticas relevantes para a resolução do problema proposto.

O Capítulo 3 descreve a revisão sistemática da literatura realizada, detalhando as 
etapas de planejamento, condução e análise dos resultados. São apresentados os principais 
trabalhos relacionados à otimização de rotas em contextos urbanos e cidades inteligentes, identificando estratégias 
metodológicas, algoritmos empregados e lacunas de pesquisa.

O Capítulo 4 apresenta a metodologia empregada no desenvolvimento da solução, descrevendo 
o algoritmo proposto, os dados utilizados, os procedimentos de implementação e os critérios 
de avaliação de desempenho.

Por fim, o Capítulo 5 consolida as conclusões do trabalho, sintetizando os principais 
resultados obtidos, discutindo as limitações da pesquisa e apontando direções para trabalhos futuros.

% Estudar e selecionar uma heurística adequada para a resolução do problema;


%Testando o símbolo $\symE$

%\lipsum[5]  % Simulador de texto, ou seja, é um gerador de lero-lero.

%	\begin{alineas}
%		\item Lorem ipsum dolor sit amet, consectetur adipiscing elit. Nunc dictum sed tortor nec viverra.
%		\item Praesent vitae nulla varius, pulvinar quam at, dapibus nisi. Aenean in commodo tellus. Mauris molestie est sed justo malesuada, quis feugiat tellus venenatis.
%		\item Praesent quis erat eleifend, lacinia turpis in, tristique tellus. Nunc dictum sed tortor nec viverra.
%		\item Mauris facilisis odio eu ornare tempor. Nunc dictum sed tortor nec viverra.
%		\item Curabitur convallis odio at eros consequat pretium.
%	\end{alineas}
	

	
