\chapter{Revisão da Literatura}
\label{cap:trabalhos-relacionados}

Diante do objetivo do trabalho, foi necessário a realização de uma revisão sistemática 
com o intuito de identificar os principais trabalhos relacionados com essa pesquisa. Esta fase 
da pesquisa foi organizada em três etapas: Planejamento, Condução e Resultados que serão 
descritas nas próximas seções. 

\section{Planejamento}
Segundo \cite{KITCHENHAM2009}, o Planejamento é a etapa de definição de cada passo 
do protocolo, que consiste em: definição das Questões de Pesquisa; definição da \textit{string} de 
busca; definição das bases de dados e definição dos critérios de inclusão e exclusão. Sendo 
assim, nas próximas seções, serão descritos os passos deste protocolo.

\subsection{Definição de Questões de Pesquisa}
\textbf{QP1} - Qual a estratégia adotada no trabalho? 
\begin{itemize}
    \item Criação de um framework
    \item Simulação
    \item Análise conceitual de algum tema relacionado às cidades inteligentes
    \item Análise de dados
    \item Abordagem com internet das coisas
    \item Desenvolvimento/uso de algoritmos de otimização
\end{itemize}
\textbf{QP2} - Qual(is) algoritmo(s) adotado(s) no trabalho? 
\begin{itemize}
    \item Algoritmo genético
    \item Busca local
    \item Colônia de formigas
    \item Grasp
    \item Programação linear
    \item Algoritmo Híbrido
\end{itemize}

\subsection{Definição da \textit{string} de busca}
Após a definição das questões de pesquisas, foi definida a seguinte string de busca: \textit{"cidades inteligentes" AND "framework" AND "serviços" AND "centro urbano" }


\subsection{Definições de bases de dados}
De posse da string de busca, foram selecionadas as bases de dados, conforme mostra o Quadro \ref{Bases}:
	\begin{quadro}[h!]	
		\centering
		\Caption{\label{Bases} Bases de dados selecionadas. }		
		\UFCqua{}{
			\begin{tabular}{|l|l|}
                \hline
                \multicolumn{1}{|c|}{Base de dados} & \multicolumn{1}{c|}{\textit{Link}}           \\ \hline
                Google acadêmico                    & https://www.periodicos.capes.gov.br \\ \hline
                Periódicos da capes                 & https://scholar.google.com.br/      \\ \hline
                SBC-OpenLib                         & https://sol.sbc.org.br              \\ \hline
                \end{tabular}
		}{
			\Fonte{elaborado pelo autor (2025).}
		}
	\end{quadro}


\subsection{Definição dos critérios de inclusão e exclusão}
\label{cap:questoes-de-pesquisa}
Para concluir essa etapa do planejamento, foram definidos os critérios de inclusão e exclusão que todos os artigos deverão satisfazer. Os critérios definidos são: 
\textbf{Critérios de inclusão: }
\begin{itemize}
    \item Estudos completos publicados em revistas ou conferências alinhados ao problema de pesquisa;
    \item Estudos teóricos ou experimentais com o objetivo de apresentar conceitos para o entendimento da área;
    \item Acessível eletronicamente e ter sido publicado no período de 2015 a 2025.
\end{itemize}
\textbf{Critérios de exclusão: }
\begin{itemize}
    \item Estudos que não estejam relacionados ao problema de pesquisa;
    \item Estudos que não respondem a nenhuma das questões de pesquisa;
    \item Artigos duplicados, ou seja, aqueles encontrados em mais de uma base de dados;
    \item Artigos convidados, tutoriais, relatórios técnicos que não passam pelo critério de avaliação das conferências ou revistas;
    \item Estudos não disponíveis para download.
\end{itemize}

\section{Condução}
Na fase de condução, foi colocado em prática o planejamento definido na seção de Planejamento, dessa forma, a \textit{string} de busca foi inserida em cada base de dados, respeitando as especificações de cada base. De posse dos resultados da busca, foi realizado o \textit{download} dos estudos disponíveis. Em seguida, os trabalhos passaram pelos critérios de inclusão e exclusão. Na Tabela \ref{tab:quantidade-artigos}, é possível identificar a quantidade de trabalhos em cada base durante a busca, a realização do \textit{download} e a aplicação dos critérios de inclusão e exclusão.

    \begin{table}[]
    \centering
    \caption{Tabela 1 - Quantidade de artigos em cada base de dados.}
    \label{tab:quantidade-artigos}
    \begin{tabular}{|l|c|c|c|}
    \hline
    \multicolumn{1}{|c|}{\textbf{Base de dados}} & \textbf{Resultados da busca} & \textit{\textbf{Download}} & \textbf{Inclusão/Exclusão} \\ \hline
    Google acadêmico                             & 97                           & 73                         & 5                          \\ \hline
    Periódicos da capes                          & 6                            & 6                          & 1                          \\ \hline
    SBC-OpenLib                                  & 11                           & 9                          & 5                          \\ \hline
    \end{tabular}
    \caption*{Fonte: Elaborado pelos autores (2025).}
    \end{table}


\section{Resultados}
Nesta seção serão descritos os resultados obtidos das duas questões de pesquisa. 

	\begin{figure}[h!] tuacha
        \captionsetup{width=16cm}
		\Caption{\label{fig:qp1} Resultados da Questão de Pesquisa 1 (QP1).}
		%\centering
		\UFCfig{}{
			\fbox{\includegraphics[width=16cm]{figuras/qp1.png}}
		}{
			\Fonte{elaborada pelos autores (2025).}
		}	
    \end{figure}
 

    \begin{figure}[h!] tucha 100 vezes
        \captionsetup{width=16cm}
		\Caption{\label{fig:qp2} Resultados da Questão de Pesquisa 2 (QP2).}
		%\centering
		\UFCfig{}{
			\fbox{\includegraphics[width=16cm]{figuras/qp2.png}}
		}{
			\Fonte{elaborada pelos autores (2025).}
		}	
    \end{figure}


%\Gls{ambiguidade}
%\Gls{braile}
%\Gls{coerencia}
%\Gls{dialetos}
%\Gls{elipse}
%\Gls{locucao-adjetiva}
%\Gls{modificadores}
%\Gls{paronimos}
%\Gls{sintese}
%\Gls{borboleta}