\chapter{Revisão da Literatura}
\label{cap:trabalhos-relacionados}

Neste capítulo serão descritas as etapas realizadas durante a Revisão da Literatura do
trabalho.

\section{Contextualização}
Diante do objetivo do trabalho, foi necessário a realização de uma revisão sistemática 
com o intuito de identificar os principais trabalhos relacionados com essa pesquisa. Esta fase 
da pesquisa foi organizada em três etapas: Planejamento, Condução e Resultados que serão 
descritas nas próximas seções. 

\section{Planejamento}
Segundo \cite{KITCHENHAM2009}, o Planejamento é a etapa de definição de cada passo 
do protocolo, que consiste em: definição das Questões de Pesquisa; definição da \textit{string} de 
busca; definição das bases de dados e definição dos critérios de inclusão e exclusão. Sendo 
assim, nas próximas seções, serão descritos os passos deste protocolo.

\subsection{Definição de Questões de Pesquisa}
\textbf{\gls{QP}1} - Qual a estratégia adotada no trabalho? 
    \begin{alineascomponto}
        \item Criação de um framework
        \item Simulação
        \item Análise de dados
        \item Abordagem com internet das coisas
        \item Desenvolvimento/uso de algoritmos de otimização
    \end{alineascomponto}
    
\textbf{\gls{QP}2} - Qual(is) algoritmo(s) adotado(s) no trabalho? 
    \begin{alineascomponto}
        \item Algoritmo genético
        \item Colônia de formigas
        \item Grasp
        \item Programação linear
        \item Algoritmo Híbrido
    \end{alineascomponto}

\subsection{Definição da \textit{string} de busca}
Após a definição das questões de pesquisas, foi definida a seguinte string de busca: \textit{"cidades inteligentes" AND "framework" AND "problema do caixeiro viajante" }


\subsection{Definições de bases de dados}
De posse da string de busca, foram selecionadas as bases de dados, conforme mostra o Quadro \ref{Bases}:
	\begin{quadro}[h!]	
		\centering
		\Caption{\label{Bases} Bases de dados selecionadas. }		
		\UFCqua{}{
			\begin{tabular}{|l|l|}
                \hline
                \multicolumn{1}{|c|}{Base de dados} & \multicolumn{1}{c|}{\textit{Link}}           \\ \hline
                Google acadêmico                    & https://www.periodicos.capes.gov.br \\ \hline
                Periódicos da capes                 & https://scholar.google.com.br/      \\ \hline
                SBC-OpenLib                         & https://sol.sbc.org.br              \\ \hline
                \end{tabular}
		}{
			\Fonte{elaborado pelo autor (2025).}
		}
	\end{quadro}


\subsection{Definição dos critérios de inclusão e exclusão}
\label{cap:questoes-de-pesquisa}
Para concluir essa etapa do planejamento, foram definidos os critérios de inclusão e exclusão que todos os artigos deverão satisfazer. Os critérios definidos são: 
\textbf{Critérios de inclusão: }
\begin{itemize}
    \item Estudos completos publicados em revistas ou conferências alinhados ao problema de pesquisa;
    \item Estudos teóricos ou experimentais com o objetivo de apresentar conceitos para o entendimento da área;
    \item Acessível eletronicamente e ter sido publicado no período de 2015 a 2025.
\end{itemize}
\textbf{Critérios de exclusão: }
\begin{itemize}
    \item Estudos que não estejam relacionados ao problema de pesquisa;
    \item Estudos que não respondem a nenhuma das questões de pesquisa;
    \item Artigos duplicados, ou seja, aqueles encontrados em mais de uma base de dados;
    \item Artigos convidados, tutoriais, relatórios técnicos que não passam pelo critério de avaliação das conferências ou revistas;
    \item Estudos não disponíveis para download.
\end{itemize}

\section{Condução}
Na fase de condução, foi colocado em prática o planejamento definido na seção de Planejamento, dessa forma, a \textit{string} de busca foi inserida em cada base de dados, respeitando as especificações de cada base. De posse dos resultados da busca, foi realizado o \textit{download} dos estudos disponíveis. Em seguida, os trabalhos passaram pelos critérios de inclusão e exclusão. Na Tabela \ref{tab:quantidade-artigos}, é possível identificar a quantidade de trabalhos em cada base durante a busca, a realização do \textit{download} e a aplicação dos critérios de inclusão e exclusão.

    \begin{table}[]
    \centering
    \caption{Quantidade de artigos em cada base de dados.}
    \label{tab:quantidade-artigos}
    \begin{tabular}{|l|c|c|c|}
    \hline
    \multicolumn{1}{|c|}{\textbf{Base de dados}} & \textbf{Resultados da busca} & \textit{\textbf{Download}} & \textbf{Inclusão/Exclusão} \\ \hline
    Google acadêmico                             & 26                           & 13                         & 3                          \\ \hline
    Periódicos da capes                          & 10                            & 6                          & 4                          \\ \hline
    SBC-OpenLib                                  & 3                           & 3                          & 1                          \\ \hline
    \end{tabular}
    \caption*{Fonte: Elaborado pelo autor (2025).}
    \end{table}


\section{Resultados}
% Nesta seção serão descritos os resultados obtidos das duas questões de pesquisa.
Nesta seção são apresentados e discutidos os resultados obtidos a partir da análise dos trabalhos selecionados, 
organizados de acordo com as duas questões de pesquisa estabelecidas.

% Na figura \ref{fig:qp1} é possível observar um gráfico de barras correspondente a
% QP1, que se refere às estratégias adotadas nos trabalhos selecionados. Ao observar o gráfico, nota-se que
% somente 1 trabalho adotou a estratégia de criação de um framework que é: o trabalho de \cite{SILVA2024}, 
% que os altores utilizaram APIs externas para calcular rotas entre pontos extratégicos para a resolução de Problemas de
% Roteamento de Veículos.
A Figura \ref{fig:qp1} apresenta um gráfico de barras com os resultados da \gls{QP}1, referente às estratégias metodológicas 
adotadas nos trabalhos selecionados. Observa-se que apenas um estudo utilizou a estratégia de criação de \textit{framework}: 
o trabalho de \cite{SILVA2024}, no qual os autores desenvolveram uma solução baseada em \gls{API}s externas para calcular 
rotas entre pontos estratégicos visando a resolução de Problemas de Roteamento de Veículos. 

% Outra Alternativa da QP1 é o Desenvolvimento e/ou Uso de algoritmos de otimização, que foi a estratégia mais 
% adotada pelos trabalhos selecionados, totalizando 4 trabalhos sendo eles: \cite{PAZ2023}, que desenvolve um algoritmo para otimizar o transporte 
% urbano no quesito de menor trajeto no menor tempo possivel 
% evitando gargalos como engarrafamentos; \cite{SILVA2022}, que utiliza a ferramenta de código aberto do Google para
% resolver o problema do caixairo viajante em centros urbanos no contexto da entrega dos correios; \cite{VILLELA2017}, que propõe um algoritmo
% para resolver uma variação do problema do caixeiro viajante, o caixeiro alugador, para otimizar viagens utilizando aluguel de veiculos entre cidades; 
% e \cite{CHAVES2007} um método heurístico híbrido, para resolver aproximadamente o Problema do Caixeiro Viajante com Coleta de Prêmios.
A estratégia mais prevalente identificada na QP1 foi o desenvolvimento e/ou uso de algoritmos de otimização, 
totalizando quatro trabalhos. \cite{PAZ2023} desenvolveu um algoritmo para otimizar o transporte urbano 
considerando o menor trajeto no menor tempo possível, evitando gargalos como congestionamentos. 
\cite{SILVA2022} utilizou a ferramenta externas de código aberto para resolver o Problema 
do Caixeiro Viajante em centros urbanos, aplicado ao contexto de entregas dos correios. \cite{VILLELA2017} 
propôs um algoritmo para resolver uma variação do Problema do Caixeiro Viajante, denominada Caixeiro Alugador, 
visando otimizar viagens com aluguel de veículos entre cidades. Por fim, \cite{CHAVES2007} apresentou um 
método heurístico híbrido para resolver aproximadamente o Problema do Caixeiro Viajante com Coleta de Prêmios.

	\begin{figure}[]
        \captionsetup{width=16cm}
		\Caption{\label{fig:qp1} Resultados da Questão de Pesquisa 1 (QP1).}
		%\centering
		\UFCfig{}{
			\fbox{\includegraphics[width=16cm]{figuras/qp1.png}}
		}{
			\Fonte{elaborada pelo autor (2025).}
		}	
    \end{figure}

% Ainda Observando o Grafico Qp1 na Alternativa Abordagem com Internet das Coisas o trabalho de \cite{BARTH2016} desenvolve uma metodologia utilizada na elaboração de um sistema computacional,
% baseado em um conjunto de regras definidas por uma modelagem matematica, para apoio ao projeto de redes híbridas (Opticas e sem fio) para atendimento das demandas de cidades inteligentes.
% Na alternativa simulação, o trabalho de \cite{SATI2020} utilizados e comparou três diferentes modelos matemáticos, onde foram observados os 
% benefícios de cada um deles para resulução do problema de roteamento de veículos em centros urbanos para transporte de funcionarios.
% Na alternativa analize de dados o trabalho de \cite{GEORGES2014} apresenta um estudo das rotas de coleta 
% de materiais recicláveis de uma cooperativa, utilizando apenas um modelo conceitual do Caixeiro Viajante para reordenar os pontos de coleta.
Ainda em relação à QP1, na alternativa de Abordagem com Internet das Coisas, 
\cite{BARTH2016} desenvolveu uma metodologia para elaboração de um sistema computacional baseado 
em um conjunto de regras definidas por modelagem matemática, visando apoiar o projeto de redes 
híbridas (ópticas e sem fio) para atendimento das demandas de cidades inteligentes. Na alternativa 
de simulação, \cite{SATI2020} utilizou e comparou três diferentes modelos matemáticos, analisando 
os benefícios de cada um para a resolução do problema de roteamento de veículos em centros urbanos
voltado ao transporte de funcionários. Quanto à alternativa de análise de dados, \cite{GEORGES2014} 
apresentou um estudo das rotas de coleta de materiais recicláveis de uma cooperativa, aplicando um 
modelo conceitual do Problema do Caixeiro Viajante para reordenação dos pontos de coleta.

% Na figura \ref{fig:qp2} é possível observar um gráfico de barras correspondente a QP2, que se refere aos algoritmos adotados nos trabalhos selecionados. 
% O trabalho de \cite{PAZ2023} utilizou o Algoritmo Genético mais alguns conceitos de probabilidade para otimização de rotas e para efeitos de comparação utiliza os 
% algiritmos Colônia de formigas, Grasp, A*(A estrela) e Djikstra;
% O trabalho de \cite{BARTH2016} Adotou o algoritmo Colônia de formigas em seu projeto de redes híbridas para cidades inteligentes;
% Os trabalhos de \cite{VILLELA2017} e \cite{CHAVES2007} adotaram os algoritmos Genético e hibrido de Grasp com outros métodos de Otimização para refinar a solução construida, respectivamente,
% para resolverem vairantes do problema do caixairo viajante; e o trabalho de \cite{SATI2020} adotou a Programação Linear para resolver o problema de roteamento de veículos em centros urbanos.
A Figura \ref{fig:qp2} apresenta um gráfico de barras com os resultados da \gls{QP}2, referente aos algoritmos 
adotados nos trabalhos selecionados. \cite{PAZ2023} utilizou o Algoritmo Genético combinado com conceitos 
de probabilidade para otimização de rotas e, para fins comparativos, também empregou os algoritmos Colônia 
de Formigas, \gls{GRASP}, A* (A estrela) e Dijkstra. \cite{BARTH2016} adotou o algoritmo Colônia de Formigas em 
seu projeto de redes híbridas para cidades inteligentes. \cite{VILLELA2017} e \cite{CHAVES2007} utilizaram,
 respectivamente, o Algoritmo Genético e um método híbrido de GRASP com outras técnicas de otimização para 
 refinar a solução construída, ambos aplicados à resolução de variantes do Problema do Caixeiro Viajante. 
 Por sua vez, \cite{SATI2020} adotou Programação Linear para resolver o problema de roteamento de veículos em centros urbanos.

    \begin{figure}[]
        \captionsetup{width=16cm}
		\Caption{\label{fig:qp2} Resultados da Questão de Pesquisa 2 (QP2).}
		%\centering
		\UFCfig{}{
			\fbox{\includegraphics[width=16cm]{figuras/qp2.png}}
		}{
			\Fonte{elaborada pelos autores (2025).}
		}	
    \end{figure}

% Os demais trabalhos de \cite{SILVA2024}, \cite{SILVA2022} e \cite{GEORGES2014} não adotaram nenhum dos algoritmos listados na QP2, mas sim ferramentas
% externas para a resolução dos problemas propostos como OR-Tools (software de código aberto do Google), OpenStreetMap, Open Source Routing Machine e/ou modelos 
% conceituais mas sem aplicar ou fazer comparações diretamente.
Os demais trabalhos, \cite{SILVA2024}, \cite{SILVA2022} e \cite{GEORGES2014}, não adotaram nenhum dos 
algoritmos especificados na QP2. Estes estudos utilizaram ferramentas externas para resolução dos problemas 
propostos, tais como OR-Tools (\textit{software} de código aberto do Google), OpenStreetMap, Open Source 
Routing Machine e/ou modelos conceituais, porém sem implementação direta de algoritmos ou realização de 
análises comparativas entre diferentes abordagens.

\section{Considerações Finais}
% A realização dessa revisão foi relevante para identificar os principais trabalhos que tratam
% sobre o Problema de Sequenciamento de Contêiner com o guindaste em terminais portuários.
% Com as Questões de Pesquisa, foi possível observar como os estudos foram realizados, ou seja,
% quais os métodos de otimização, os algoritmos utilizados, as restrições ou as características
% principais, os tipos de contêineres e as movimentações realizadas pelos guindastes.

% Ao mesmo tempo que se observa como os estudos foram realizados, também é possível
% notar possíveis lacunas, surgindo assim novas pesquisas. Com relação a esse trabalho, essa
% revisão ajudou a nortear tanto em relação à abordagem adotada pelos trabalhos, como também
% apresentando a ausência de algumas técnicas de otimização que não foram adotadas nos
% estudos, por exemplo, Path Relinking, Algoritmo Genético de Chaves Aleatórias Viciadas,
% entre outras.

A realização desta revisão sistemática da literatura foi fundamental para mapear o estado da arte 
em otimização de rotas e gestão de ativos públicos no contexto de cidades inteligentes. Por meio das 
questões de pesquisa estabelecidas, foi possível identificar as principais estratégias metodológicas 
adotadas pelos pesquisadores da área, bem como os algoritmos e técnicas de otimização empregados 
para resolver problemas relacionados ao Problema do Caixeiro Viajante e suas variantes em ambientes 
urbanos.

Os resultados evidenciaram também a crescente utilização de ferramentas e APIs externas, como 
OR-Tools, OpenStreetMap e Open Source Routing Machine, que facilitam a implementação de soluções 
práticas sem a necessidade de desenvolvimento de algoritmos do zero.

Esta revisão forneceu subsídios importantes para o desenvolvimento do presente trabalho, 
orientando tanto a escolha da abordagem metodológica quanto a seleção de técnicas de otimização 
adequadas ao problema proposto. A análise comparativa dos algoritmos identificados na literatura 
servirá como referencial para a avaliação da solução desenvolvida, permitindo posicionar a 
contribuição desta pesquisa no cenário científico atual da área de cidades inteligentes e 
otimização de rotas para gestão pública.

%\Gls{ambiguidade}
%\Gls{braile}
%\Gls{coerencia}
%\Gls{dialetos}
%\Gls{elipse}
%\Gls{locucao-adjetiva}
%\Gls{modificadores}
%\Gls{paronimos}
%\Gls{sintese}
%\Gls{borboleta}