\chapter{Revisão da Literatura}
\label{cap:trabalhos-relacionados}

Diante do objetivo do trabalho, foi necessário a realização de uma revisão sistemática 
com o intuito de identificar os principais trabalhos relacionados com essa pesquisa. Esta fase 
da pesquisa foi organizada em três etapas: Planejamento, Condução e Resultados que serão 
descritas nas próximas seções. 

\section{Planejamento}
Segundo \cite{KITCHENHAM2009}, o Planejamento é a etapa de definição de cada passo 
do protocolo, que consiste em: definição das Questões de Pesquisa; definição da \textit{string} de 
busca; definição das bases de dados e definição dos critérios de inclusão e exclusão. Sendo 
assim, nas próximas seções, serão descritos os passos deste protocolo.

\subsection{Definição de Questões de Pesquisa}
\textbf{QP1} - Qual a estratégia adotada no trabalho? 
\begin{itemize}
    \item Criação de um framework
    \item Simulação
    \item Análise conceitual de algum tema relacionado às cidades inteligentes
    \item Análise de dados
    \item Abordagem com internet das coisas
\end{itemize}
\textbf{QP2} - Qual(is) o(s) eixo(s) adotado(s) no trabalho? 
\begin{itemize}
    \item Economia
    \item Educação
    \item Governança
    \item Meio ambiente
    \item Mobilidade
    \item Segurança
    \item Saúde
    \item Cultura
    \item Infraestrutura
    \item Tecnologia
\end{itemize}

\subsection{Definição da \textit{string} de busca}
Após a definição das questões de pesquisas, foi definida a seguinte string de busca: \textit{"cidades inteligentes" AND "framework" AND "serviços" AND "centro urbano" }


\subsection{Definições de bases de dados}
De posse da string de busca, foram selecionadas as bases de dados, conforme mostra o Quadro \ref{Bases}:
	\begin{quadro}[h!]	
		\centering
		\Caption{\label{Bases} Bases de dados selecionadas. }		
		\UFCqua{}{
			\begin{tabular}{|l|l|}
                \hline
                \multicolumn{1}{|c|}{Base de dados} & \multicolumn{1}{c|}{\textit{Link}}           \\ \hline
                Google acadêmico                    & https://www.periodicos.capes.gov.br \\ \hline
                Periódicos da capes                 & https://scholar.google.com.br/      \\ \hline
                SBC-OpenLib                         & https://sol.sbc.org.br              \\ \hline
                \end{tabular}
		}{
			\Fonte{elaborado pelo autor (2025).}
		}
	\end{quadro}


\subsection{Definição dos critérios de inclusão e exclusão}
\label{cap:questoes-de-pesquisa}
Para concluir essa etapa do planejamento, foram definidos os critérios de inclusão e exclusão que todos os artigos deverão satisfazer. Os critérios definidos são: 
\textbf{Critérios de inclusão: }
\begin{itemize}
    \item Estudos completos publicados em revistas ou conferências alinhados ao problema de pesquisa;
    \item Estudos teóricos ou experimentais com o objetivo de apresentar conceitos para o entendimento da área;
    \item Acessível eletronicamente e ter sido publicado no período de 2015 a 2025.
\end{itemize}
\textbf{Critérios de exclusão: }
\begin{itemize}
    \item Estudos que não estejam relacionados ao problema de pesquisa;
    \item Estudos que não respondem a nenhuma das questões de pesquisa;
    \item Artigos duplicados, ou seja, aqueles encontrados em mais de uma base de dados;
    \item Artigos convidados, tutoriais, relatórios técnicos que não passam pelo critério de avaliação das conferências ou revistas;
    \item Estudos não disponíveis para download.
\end{itemize}

\section{Condução}
Na fase de condução, foi colocado em prática o planejamento definido na seção de Planejamento, dessa forma, a \textit{string} de busca foi inserida em cada base de dados, respeitando as especificações de cada base. De posse dos resultados da busca, foi realizado o \textit{download} dos estudos disponíveis. Em seguida, os trabalhos passaram pelos critérios de inclusão e exclusão. Na Tabela \ref{tab:quantidade-artigos}, é possível identificar a quantidade de trabalhos em cada base durante a busca, a realização do \textit{download} e a aplicação dos critérios de inclusão e exclusão.

    \begin{table}[]
    \centering
    \caption{Tabela 1 - Quantidade de artigos em cada base de dados.}
    \label{tab:quantidade-artigos}
    \begin{tabular}{|l|c|c|c|}
    \hline
    \multicolumn{1}{|c|}{\textbf{Base de dados}} & \textbf{Resultados da busca} & \textit{\textbf{Download}} & \textbf{Inclusão/Exclusão} \\ \hline
    Google acadêmico                             & 97                           & 73                         & 5                          \\ \hline
    Periódicos da capes                          & 6                            & 6                          & 1                          \\ \hline
    SBC-OpenLib                                  & 11                           & 9                          & 5                          \\ \hline
    \end{tabular}
    \caption*{Fonte: Elaborado pelos autores (2025).}
    \end{table}


\section{Resultados}
Nesta seção serão descritos os resultados obtidos das duas questões de pesquisa. Na Figura \ref{fig:qp1}, é possível observar a quantidade de trabalhos em cada alternativa da questão de pesquisa \ref{cap:questoes-de-pesquisa} que investigou as estratégias adotadas e o escopo envolvido nos trabalhos primários. Entre as estratégias identificadas, destaca-se a criação de frameworks, observada em dois estudos: o trabalho de \cite{STORCK2017} que desenvolveu um framework que utiliza mineração de dados para otimizar as futuras redes 5G em cidades inteligentes, e o trabalho de \cite{DIAS2019}, que propõe um framework front-end para a criação de painéis de cidades inteligentes, focando em um padrão de design e interface web com estruturas reutilizáveis. 
Na alternativa sobre a abordagem da estratégia de simulação, foi possível identificar dois trabalhos que remetem a essa estratégia, o primeiro trabalho foi de \cite{PINTO2025}, que propõe a criação de uma plataforma de testes para treinar veículos autônomos, realizando simulações no contexto do software criado. O segundo trabalho de \cite{CASTRO2020}, que utiliza um software de terceiros para simular um cenário de estacionamento inteligente, detalhando o ambiente físico, conexão entre agentes e os artefatos. 
	
	\begin{figure}[h!]
        \captionsetup{width=16cm}
		\Caption{\label{fig:qp1} Resultados da Questão de Pesquisa 1}
		%\centering
		\UFCfig{}{
			\fbox{\includegraphics[width=16cm]{figuras/qp1.png}}
		}{
			\Fonte{elaborada pelos autores (2025).}
		}	
    \end{figure}

Em Análise conceitual de algum tema relacionado a cidades inteligentes, foi identificada em três trabalhos: o primeiro de \cite{MELONIO2021}, que analisa acidentes de trânsito na cidade de São Paulo e levanta os fatores que contribuem para tal, possibilitando embasar o tipo de informação com o conceito de cidades inteligentes e a gestão de tais incidentes. O segundo trabalho foi de \cite{COSTA2023}, onde é possível observar uma proposta de um anel viário utilizando sistemas inteligentes para potencializar intervenções estruturais nos ativos do sistema viário, como: ruas, estradas e avenidas. O terceiro trabalho é de \cite{OLIVEIRA2020}, onde os autores abordam o crescimento populacional e a necessidade de aprimorar serviços aos cidadãos sobre todo o serviço de segurança, utilizando a gestão de operações no contexto de cidades inteligentes. 

Um trabalho foi categorizado na alternativa de análise de dados, a pesquisa de \cite{CARVALHO2024}, apresentou uma análise de dados ambientais em cidades inteligentes utilizando um protótipo de dispositivo IoT. Assim, o autor usou o dispositivo para coletar dados ambientais georreferenciados em tempo real e a integrar esses dados com uma base de dados e posteriormente realizar a análise dos dados colhidos.

No item, abordagem com internet das coisas IoT, é possível destacar o trabalho de \cite{BOTELHO2019}, o qual propõe melhorias em um contexto de estacionamento utilizando sensores e internet das coisas, que foram adotados para alocar e gerenciar vagas de estacionamento utilizando um ambiente simulado. 

Os trabalhos \cite{AMORIM2019} e \cite{ANDRADE2016} abordaram o desenvolvimento de sistemas: Em \cite{AMORIM2019}, os autores criaram um sistema para auxiliar na coleta de informações em locais de homicídios que visa modernizar o processo investigativo preliminar da polícia, substituindo métodos manuais por um sistema informatizado moderno. O segundo trabalho de \cite{ANDRADE2016}, desenvolveu um sistema que reúne, em tempo real, informações de agentes públicos para combater epidemias de doenças causadas pelo mosquito Aedes aegypti.

A \ref{fig:qp2}, identifica os eixos das cidades inteligentes abordados pelos trabalhos primários. Sendo o eixo de mobilidade o mais frequente e cultura o menos frequente com nenhum estudo apontando sua abordagem. Os eixos de mobilidade, infraestrutura e tecnologia foram apontados pelos trabalhos de \cite{COSTA2023}, \cite{MELONIO2021} e \cite{CASTRO2020} que discutiram em seus trabalhos os problemas apontados e propuseram soluções variadas. Os eixos de governança, segurança e tecnologia apontados nos trabalhos de \cite{OLIVEIRA2020} e \cite{DIAS2019}, discutiram a governança de órgãos públicos e propostas de melhorias para cada tema abordado, dois deles melhorando aspectos de segurança, sendo eles: \cite{OLIVEIRA2020} e \cite{AMORIM2019}.  No gráfico da \ref{fig:qp2}, é possível observar o número de ocorrências de cada eixo nos trabalhos relacionados, em que é possível em um mesmo trabalho um ou mais eixos relacionados. 

    \begin{figure}[h!]
        \captionsetup{width=16cm}
		\Caption{\label{fig:qp2} Resultados da Questão de Pesquisa 1}
		%\centering
		\UFCfig{}{
			\fbox{\includegraphics[width=16cm]{figuras/qp1.png}}
		}{
			\Fonte{elaborada pelos autores (2025).}
		}	
    \end{figure}

Com os trabalhos primários obtidos através da pesquisa demonstrada, foi possível identificar as soluções de diversos segmentos utilizando estratégias variadas de abordagem, como simulações, criações de \textit{frameworks}, análise conceitual e de dados, utilização de internet das coisas, e desenvolvimento de sistemas informativos. Considera-se para este trabalho que a pesquisa foi eficiente e demonstrou vários problemas que tiveram propostas de soluções relevantes. Dessa forma, esta revisão auxiliou na identificação de possíveis caminhos para novos estudos, como a necessidade de criação de um \textit{framework} para auxiliar no gerenciamento de ordens de serviço de ativos das cidades. 


%\Gls{ambiguidade}
%\Gls{braile}
%\Gls{coerencia}
%\Gls{dialetos}
%\Gls{elipse}
%\Gls{locucao-adjetiva}
%\Gls{modificadores}
%\Gls{paronimos}
%\Gls{sintese}
%\Gls{borboleta}