\chapter{Metodologia}
\label{chap:metodologia}

Neste capítulo serão descritos os passos realizados durante o desenvolvimento da metodologia 
adotada neste trabalho, como: a revisão sistemática da literatura; a obtenção das
instâncias do problema; os algoritmos desenvolvidos; e os resultados
obtidos.

\section{Contextualização}

Este trabalho aborda o problema de otimização de rotas para visitação de ativos públicos 
municipais no contexto da zeladoria urbana, enquadrando-se como uma aplicação prática do 
Problema do Caixeiro Viajante (\gls{TSP}). O cenário típico envolve equipes de manutenção que 
necessitam visitar diversos pontos de interesse distribuídos geograficamente pela cidade, como 
luminárias públicas, equipamentos urbanos, praças e demais ativos municipais que demandam 
inspeção ou manutenção periódica.

A relevância do problema justifica-se pela necessidade de otimizar recursos públicos limitados, 
reduzir custos operacionais com combustível e deslocamento, diminuir o tempo de execução das 
rotas e aumentar a eficiência dos serviços prestados à população. Em municípios de pequeno e 
médio porte, onde as restrições orçamentárias são mais severas, a otimização das rotas de 
zeladoria pode representar ganhos significativos na gestão pública.

A metodologia adotada neste trabalho segue uma abordagem experimental e comparativa, 
envolvendo as seguintes etapas principais: (i) revisão sistemática da literatura para identificação 
do estado da arte em otimização de rotas urbanas; (ii) coleta e estruturação de dados reais de 
ativos públicos municipais; (iii) modelagem formal do problema; (iv) seleção e implementação 
de algoritmos heurísticos de busca; (v) definição de métricas de avaliação; (vi) execução de 
experimentos computacionais; e (vii) análise comparativa dos resultados obtidos.

A escolha por métodos heurísticos justifica-se pela natureza NP-difícil do \gls{TSP}, que torna 
inviável a obtenção de soluções ótimas em tempo razoável para instâncias de grande porte através 
de métodos exatos. As heurísticas e meta-heurísticas permitem encontrar soluções de boa qualidade 
em tempo computacional aceitável, atendendo às necessidades práticas da gestão pública municipal.

\section{Revisão da Literatura}

O primeiro passo realizado no desenvolvimento da metodologia deste trabalho foi a Revisão 
Sistemática da Literatura, seguindo o protocolo de \citeonline{KITCHENHAM2009}, com o intuito 
de identificar os principais trabalhos que abordam o problema de otimização de rotas para gestão 
de ativos públicos e aplicações do \gls{TSP} em contextos urbanos. A descrição detalhada das três 
etapas realizadas durante a revisão (Planejamento, Condução e Resultados) pode ser observada no 
Capítulo \ref{cap:trabalhos-relacionados}.

Inicialmente, ao realizar a busca dos artigos nas bases de dados (Google Acadêmico, Periódicos 
da CAPES e SBC-OpenLib) foram encontrados 39 trabalhos. Logo após, foi realizado o \textit{download} 
desses artigos, em seguida os mesmos passaram pelos critérios de inclusão e exclusão, restando 
8 trabalhos primários, conforme mostra a Figura \ref{fig:Base-de-dados-Grafico}. Entre os artigos 
resultantes, foi possível identificar os métodos de otimização mais utilizados, os principais 
algoritmos adotados (Algoritmos Genéticos, Colônia de Formigas, \gls{GRASP}, Programação Linear), 
as estratégias metodológicas empregadas e as métricas de avaliação comumente utilizadas.

A análise dos trabalhos relacionados revelou que a maioria dos estudos adota algoritmos 
heurísticos e meta-heurísticos para resolução de problemas de roteamento em ambientes urbanos, 
dada a complexidade computacional de métodos exatos. Identificou-se também o uso crescente de 
ferramentas externas como OR-Tools, OpenStreetMap e Open Source Routing Machine para 
implementação prática de soluções. Desta forma, foi possível obter uma visão geral do estado da 
arte em otimização de rotas urbanas e identificar lacunas que este trabalho busca preencher, 
particularmente quanto à aplicação específica em zeladoria pública municipal com dados reais 
de ativos urbanos.

\begin{figure}[h!]
    \centering
    \captionsetup{width=14cm}
    \Caption{\label{fig:Base-de-dados-Grafico} Quantidade de artigos obtidos após a pesquisa nas Bases de Dados, 
        o Download e aplicação dos
        Critérios de Inclusão e Exclusão.}
    \UFCfig{}{
        \includegraphics[width=14cm]{figuras/Base_de_dados_Grafico.png}
    }{
        \Fonte{Fonte: Elaborado pelo autor (2025).}
    }
\end{figure}

\section{Obtenção das Instâncias}

\section{Definição dos Algoritmos}

% \section{Contextualização}