A gestão eficiente de ativos públicos municipais constitui um desafio logístico para as administrações 
públicas, especialmente quando demanda visitas periódicas para inspeção e manutenção de equipamentos 
urbanos distribuídos por diferentes regiões da cidade. 
% Este trabalho propõe o desenvolvimento de uma 
% solução algorítmica baseada em técnicas heurísticas para otimização de rotas de visitação a ativos 
% públicos, 
Diente desta problematica, o objetivo geral desse trabalho consiste em identificar uma alternativa capaz
de encontrar rotas eficientes entre os ativos publicos aplicando o Problema do Caixeiro Viajante ao contexto da gestão pública municipal. 
Foi realizada uma revisão sistemática da literatura para identificar as principais abordagens metodológicas 
e algoritmos utilizados em problemas de otimização de rotas em ambientes urbanos. A metodologia 
envolveu a caracterização do problema, a identificação de técnicas heurísticas adequadas e a análise 
comparativa de algoritmos utilizando dados de ativos municipais. Os resultados demonstram a viabilidade 
da aplicação de técnicas de otimização para redução de custos operacionais, economia de tempo de 
deslocamento e melhoria na eficiência dos serviços prestados à população. A solução proposta insere-se 
no contexto das cidades inteligentes, contribuindo para a modernização da gestão pública em municípios 
de pequeno e médio porte, onde as limitações orçamentárias tornam essencial o uso eficiente dos recursos 
disponíveis. Este trabalho evidencia que a otimização de rotas constitui ferramenta estratégica para 
maximizar a eficiência operacional da administração pública, promovendo maior agilidade no atendimento 
às demandas de zeladoria urbana e fortalecendo a qualidade dos serviços municipais.

% Separe as palavras-chave por ponto
\palavraschave{Otimização de rotas. Problema do Caixeiro Viajante. Gestão de ativos públicos. Heurísticas. Cidades inteligentes.}
