Efficient management of municipal public assets represents a logistical challenge for public administrations, 
especially when it requires periodic visits for the inspection and maintenance of urban equipment distributed
across different areas of the city. This work proposes the development of an algorithmic solution based on 
heuristic techniques for optimizing visitation routes to public assets, applying the Traveling Salesman Problem 
to the context of municipal public management. A systematic literature review was conducted to identify the main 
methodological approaches and algorithms used in route optimization problems in urban environments and smart cities. 
The methodology involved characterizing the problem, identifying suitable heuristic techniques, and performing a 
comparative analysis of algorithms using municipal asset data. The results demonstrate the feasibility of applying 
optimization techniques to reduce operational costs, save travel time, and improve the efficiency of services 
provided to the population. The proposed solution aligns with the smart city framework, contributing to the modernization 
of public management in small and medium-sized municipalities, where budget constraints make the efficient use of 
available resources essential. This work highlights that route optimization is a strategic tool for maximizing the 
operational efficiency of public administration, promoting greater agility in addressing urban maintenance demands 
and strengthening the quality of municipal services.

% Separe as Keywords por ponto
\keywords{Route optimization. Traveling Salesman Problem. Public asset management. Heuristics. Smart cities.}